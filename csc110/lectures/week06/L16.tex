\documentclass[14pt]{article}
\usepackage[utf8]{inputenc}
\usepackage{amsmath}
\usepackage{amssymb}
\usepackage{amsfonts}
\usepackage{amsthm}
\usepackage{mathtools}

\usepackage[top=0.75in, bottom=0.75in,left=1.8in, right=1.8in]{geometry}

\setlength{\parindent}{0pt}
\setlength{\parskip}{5pt}

\title{CSC110 Lecture 16 Starter File}
\author{}
\date{}

\begin{document}
\maketitle

\newcommand{\N}{\mathbb{N}}
\newcommand{\Z}{\mathbb{Z}}
\newcommand{\floor}[1]{\lfloor #1 \rfloor}

\section*{Definitions}

$$d \mid n:~ \exists k \in \Z,~ n = dk \qquad \text{where $d, n \in \Z$}$$
\begin{align*}
& IsPrime(p): \\
& \quad p > 1 \land \big(\forall d \in \N,~ d \mid p \Rightarrow d = 1 \lor d = p\big) \\
& \quad \text{where $p \in \Z$}
\end{align*}

\section*{Demo 1}
Prove that: $23 \mid 115$

\begin{proof}
% TODO: Complete the proof
\end{proof}

\section*{Demo 2}
Prove that there exists an integer that divides 104.

\begin{proof}
% TODO: Complete the proof
\end{proof}

\newpage
\section*{Demo 3}
Prove that every integer is divisible by 1.

\textbf{Translation}.

$$\forall n \in \Z,~ 1 \mid n$$

\textbf{Unpack definition}.

$$\forall n \in \Z,~ \exists k \in \Z,~ n = 1 \cdot k$$

\begin{proof}
% TODO: Complete the proof
\end{proof}

\newpage
\section*{Demo 4}

Prove that for all integers $x$,
if $x$ divides $x + 5$ then $x$ also divides 5.

\textbf{Translation}.

$$\forall x \in \Z,~ x \mid x + 5 \Rightarrow x \mid 5$$

\textbf{Unpack definition}.

$$
\forall x \in \Z,~
(\exists k_1 \in \Z,~ x + 5 = k_1 x) \Rightarrow
(\exists k_2 \in \Z,~ 5 = k_2 x)
$$

\begin{proof}
% TODO: complete the proof
\end{proof}

\textbf{Rough Work}.
% TODO: complete rough work to help complete the proof above

\newpage
\section*{Exercise 1: Practice with Proofs}

1. Prove the following statement using the defintion of divisibility.

$$\forall n, d, a \in \Z,~ d \mid n \Rightarrow d \mid an$$

\begin{proof}
% TODO: complete the proof
\end{proof}

\newpage

2. Consider this statement.

$$\forall n, d, a \in \Z,~ d \mid an \Rightarrow d \mid a \lor d \mid n$$

This statement is false.



\textbf{Negation}.

First, write the negation of this statement.
Tip: in Latex, \verb|\nmid| means "does not divide"

% TODO: write the negation

\textbf{Proving the negation.}

Prove the negation of the statement. (By proving the statement’s negation is True, you’ll prove that the original statement is False.)

\begin{proof}
% TODO: complete the proof
\end{proof}

\newpage
\section*{Exercise 2: Primality Testing}

In lecture, we saw an algorithm for checking whether a number $p$ is prime that checks all of the possible factors of $p$ between $2$ and $\floor{\sqrt{p}}$, inclusive.

We can prove that this algorithm is correct by proving the following statement:

$$\forall p \in \Z,~ \mathit{Prime}(p) \Leftrightarrow \big(p > 1 \land (\forall d \in \N,~ 2 \leq d \leq \sqrt{p} \Rightarrow d \nmid p) \big).$$

This is a larger statement than the ones we've looked at so far, so on this exercise we've broken down the proof of this statement for you to complete.

\begin{proof}
Let $p \in \Z$. We need to prove an if and only if, which we do by dividing the proof into two parts.

\textbf{Part 1}: Prove that $\mathit{Prime}(p) \Rightarrow \big(p > 1 \land (\forall d \in \N,~ 2 \leq d \leq \sqrt{p} \Rightarrow d \nmid p) \big)$.

% 1. Write down what we can **assume** in this part of the proof.

% 2. To prove an AND, we need to prove that both parts are true. First, prove that $p > 1$.

% 3. Now, prove that $\forall d \in \N,~ 2 \leq d \leq \sqrt{p} \Rightarrow d \nmid p$.


\textbf{Part 2}: Prove that $p > 1 \land (\forall d \in \N,~ 2 \leq d \leq \sqrt{p} \Rightarrow d \nmid p) \Rightarrow \mathit{Prime}(p)$.

% 4. Write down what we can **assume** in this part of the proof.

% We need to prove that $\mathit{Prime}(p)$, which expands into $p > 1 \land (\forall d \in \N,~ d \mid p \Rightarrow d = 1 \lor d = p)$.

% 5. First, prove that $p > 1$.

% 6. Now for the proof of $\forall d \in \N,~ d \mid p \Rightarrow d = 1 \lor d = p$. Start by writing the appropriate proof header, introducing the variable $d$ and assumption about $d$.

% 7. Use the **contrapositive** of a part of your original assumption. What can you conclude about $d$?

% 8. Using the cases from the previous part, prove that $d = 1 \lor d = p$.

\end{proof}

\end{document}
