\documentclass[fontsize=11pt]{article}
\usepackage{amsfonts}
\usepackage{amsmath}
\usepackage{amsthm}
\usepackage[utf8]{inputenc}
\usepackage[margin=0.75in]{geometry}

\title{CSC110 Assignment 4: Number Theory, Cryptography, and Algorithm Running Time}
\author{Jamie Yi}
\date{\today}

% Some useful LaTeX commands. You are free to use these or not, and also add your own.
\newcommand{\N}{\mathbb{N}}
\newcommand{\Z}{\mathbb{Z}}
\newcommand{\R}{\mathbb{R}}
\newcommand{\cO}{\mathcal{O}}
\newcommand{\floor}[1]{\left\lfloor #1 \right\rfloor}

\begin{document}
\maketitle

\section*{Part 1: Practice with Proofs}

\begin{enumerate}

\item[1.] Statement to prove:

$$\forall a, k, n \in \Z,~ \gcd(a, n) = 1 \Rightarrow \gcd(a + kn, n) = 1$$

\begin{proof}
: \\ -Let $e \in \N$ \\
-Asusume $e|a + kn$, this implies: $\exists k_1 \in \Z$ s.t. $a + kn = k_1 e$ \\
-Asusume $e|n$, this implies: $\exists k_2 \in \Z$ s.t. $n = k_2 e$ \\~\\
-We can rearrange $a + kn = k_1 e$ to be $a = k_1 e - kn$ \\
-Substitute in $n = k_2 e$, to get $a = k_1 e - k(k_2 e) = e(k_1 - kk_2)$ \\
-Thus, $e|a$\\~\\
-Since $e|n$(by assumption) and $e|a$ and $\gcd(a, n) = 1$, $e \leq 1$ \\
-Since $ e = \gcd(a + kn, n)$ and $e \leq 1$ and 1 divides all numbers, therefore $\gcd(a + kn, n) = 1$ as needed

\end{proof}

\item[2.] Statement to prove (we've expanded the definition of Omega for you!):

$$\exists c, n_0 \in \R^+,~ \forall n \in \N,~ n \geq n_0 \Rightarrow \log_{3} n - \log_{11} n \geq c \cdot \log_{14} n$$

\begin{proof}
: \\ -let $c = 1.2$ \\
-Let $n_0 = 1$ \\
-We know $2.4 < \frac{1}{\log_{14} 3} < 2.5$ and $1.1 < \frac{1}{\log_{14} 3} < 1.2$ \\~\\
-Since $n \geq 1$, $\log_{14} n$, $\log_{3} n$, and $\log_{11} n$ will never be negative \\
$log_{3} n - log_{11} n$ \\
$= \frac{\log_{14} n}{\log_{14} 3} - \frac{\log_{14} n}{\log_{14} 11}$ \\
$= \log_{14} n\frac{1}{\log_{14} 3} - \log_{14} n \frac{1}{\log_{14} 11}$ \\
$= \log_{14} n\frac{1}{\log_{14} 3} + (-\log_{14} n \frac{1}{\log_{14} 11})$ \\
$\geq 2.4\log_{14}c+ (-1.2\log_{14} n)$ \\
$= 1.2\log_{14} n$ \\~\\
-Therefore $\log_{3} n - \log_{11} n \geq c \cdot \log_{14} n$ as needed

\end{proof}

\item[3.] Statement to prove (we haven't expanded the definition of Big-O for you, but we encourage you to do so yourself):

$$\forall f, g: \N \to \R^{\geq 0},~ g \in \cO(f) \land \big(\forall m \in \N,~ f(m) \geq 1 \big) \Rightarrow g \in \cO(\floor{f})$$

\begin{proof}
Definition of $g \in \cO(f)$ (our assumption): $$\exists c_1, n_1 \in \R^+,~ \forall n \in \N,~ n \geq n_1 \Rightarrow g(n) \leq c_1 \cdot f(n)$$

Definition of $g \in \cO(\floor{f})$ what we WTS: $$\exists c_2, n_2 \in \R^+,~ \forall n \in \N,~ n \geq n_2 \Rightarrow g(n) \leq c_2 \cdot \floor{f(n)}$$

-Take $c_2 = 2c_1$ \\
-Take $n_2 = n_1$ \\
-Using what we know that $ x < \floor{n} + 1$, we know that $ c_1 \cdot n < c_1 \cdot \floor{n} + c_1$ \\
-We also know that $\floor{m} \geq 1$(for all $m \in \N$) \\~\\
$g(n) \leq c_1 \cdot f(n)$ (by assumption) \\
$~~~~~~~ < c_1 \cdot \floor{f(n)} + c_1$ \\
$~~~~~~~ \leq c_1 \cdot \floor{f(n)} + c_1 \cdot \floor{f(n)}$ (The inequality holds since we know $\floor{f(x)} \geq 1$) \\
$~~~~~~~ = 2c_1 \cdot \floor{f(n)}$ \\
$~~~~~~~ = c_2 \cdot \floor{f(n)}$ as needed

\end{proof}

\end{enumerate}

\newpage

\section*{Part 2: Generating Coprime Numbers}

\begin{enumerate}

\item[1.]
Not to be handed in.

\item[2.]
Complete this part in the provided \texttt{a4\_part2.py} starter file.
Do \textbf{not} include your solution in this file.

\item[3.]
Prove that each loop invariant holds.

\begin{enumerate}
\item[a.] Loop Invariant 1
\begin{proof}
: \\-Let $i_0$ be \texttt{len(nums\_so\_far) - 1} at the start of the iteration \\
-Take $i = i_0 + 1$ \\~\\
We know at the start of the iteration, for all $k$ in \texttt{nums\_so\_far}, $\gcd(k, 2) = 1$ and $\gcd(k, 3) = 1$. \\
Looking at the first line in the while, we can also tell that $k$ can be written as $k = 1 + 6e$ or $k = 5 + 6e$ where $e$ is some $\Z$ ($e$ does not need to be the same for every instance of $k$) \\~\\
Case 1: \texttt{nums\_so\_far[$i_0$]} $= 1 + 6e$, where $ e \in \Z$ \\
-In this case, at the end of the iteration, \texttt{nums\_so\_far[$i$]} $= 5 + 6e$ \\
-We know $\gcd(5, 2) = 1$, so by Part 1 Question 1 we know $\gcd(5 + 2(3e), 2) = 1 $ \\
-We know $\gcd(5, 3) = 1$, so by Part 1 Question 1 we know $\gcd(5 + 3(2e), 3) = 1 $ \\~\\

Case 2: \texttt{nums\_so\_far[$i_0$]} $= 5 + 6e$, where $ e \in \Z$ \\
-In this case, at the end of the iteration, \texttt{nums\_so\_far[$i$]} $= 7 + 6e$, or $= 1 + 6(e + 1)$  \\
-We know $\gcd(1, 2) = 1$, so by Part 1 Question 1 we know $\gcd(1 + 2(3 \cdot (e + 1), 2) = 1 $ \\
-We know $\gcd(1, 3) = 1$, so by Part 1 Question 1 we know $\gcd(1 + 3(2 \cdot (e + 1), 3) = 1 $ \\~\\

Therefore at the end of the iteration, every number $k$ in \texttt{nums\_so\_far} is coprime to 2 and coprime to 3.
\end{proof}

\item[b.] Loop Invariant 2
\begin{proof}
: \\-Let $i_0$ be \texttt{len(nums\_so\_far)} $- ~3$ at the start of the iteration \\
-Take $i = i_0 + 1$ \\~\\

We know for all $j$ between 0 and \texttt{len(nums\_so\_far)} $- ~2$, \texttt{nums\_so\_far[$i_0$]} $ + ~6 =$  \texttt{nums\_so\_far[$i_0 + 2$]}. \\
The first two lines of the iteration, we're appedning the value \texttt{nums\_so\_far[$i_0 + 1$]} $ + ~6$ to the end of the list (\texttt{nums\_so\_far[$i_0 + 1$]} is the same index as \texttt{nums\_so\_far[$-2$]}, they both refer to the second last element of the list). The value is appended to index \texttt{nums\_so\_far[$i_0 + 3$]} (increasing the length of the list). \\~\\

Thus, \texttt{nums\_so\_far[$i_0 + 1$]} $ + ~6 =$ \texttt{nums\_so\_far[$i_0 + 3$]} \\
This can be rewritten as \texttt{nums\_so\_far[$i$]} $ + ~6 =$ \texttt{nums\_so\_far[$i + 2$]} \\
Now that the length of the list has increased, $i$ now represents \texttt{len(nums\_so\_far) - 3}\\~\\

Therefore, for all natural numbers $j$ between 0 and \texttt{len(nums\_so\_far)} $- ~3$ inclusive, \\ \texttt{nums\_so\_far[$i$]} $+ ~6$ = \texttt{nums\_so\_far[$i + 2$]} by the end of the iteration.
\end{proof}

\item[c.] Loop Invariant 3
\begin{proof}
: \\-Let $i_0$ be \texttt{len(nums\_so\_far)} $- ~2$ at the start of the iteration \\
-Take $i = i_0 + 1$ \\~\\

We know \texttt{nums\_so\_far[$i_0$]} $<$ \texttt{nums\_so\_far[$i_0 + 1$]} at the start of the iteration. By loop invariant 2, we know the value \texttt{nums\_so\_far[$i_0$]} $+ ~6$ will be appended to the list at index \texttt{nums\_so\_far[$i_0 + 2$]}. So far, we know that \texttt{nums\_so\_far[$i_0$]} $+ ~6$ = \texttt{nums\_so\_far[$i_0 + 2$]}. \\
Seeing as how \texttt{nums\_so\_far} is initially $[1, 5]$, we can deduce that for any two consecutive elements in the list, \texttt{nums\_so\_far[$j + 1$]} - \texttt{nums\_so\_far[$j$]} $\leq 4$. We can deduce this knowing that the difference between 5 and 1 is 4, and we're only adding to previous lists by a constant 6, and that the difference between the constant 6 and the initial difference 4 is 2. Thus consecutive elements will always be increasing, and either have a difference of 4 or 2. \\
From this we can say that \texttt{nums\_so\_far[$i_0$]} $<$ \texttt{nums\_so\_far[$i_0 + 1$]} $\leq$ \texttt{nums\_so\_far[$i_0$]} $+ ~4$ \\
This implies \texttt{nums\_so\_far[$i_0$]} $<$ \texttt{nums\_so\_far[$i_0 + 1$]} $<$ \texttt{nums\_so\_far[$i_0$]} $+ ~6$ \\
Which can be simplified as \texttt{nums\_so\_far[$i_0 + 1$]} $<$ \texttt{nums\_so\_far[$i_0 + 2$]} (by loop invariant 2)\\
Which can be rewritten as \texttt{nums\_so\_far[$i$]} $<$ \texttt{nums\_so\_far[$i + 1$]} \\~\\

Therefore, for all natural numbers $j$ between 0 and \texttt{len(nums\_so\_far)} $- ~2$ inclusive, \texttt{len(nums\_so\_far[$j$])} $<$ \texttt{len(nums\_so\_far[$j + 1$])} (this means that \texttt{nums\_so\_far} is always sorted).
\end{proof}

\item[d.] Loop Invariant 4
\begin{proof}
: \\-Let $i_0$ be \texttt{len(nums\_so\_far)} $- ~1$ at the start of the iteration \\
-Take $i = i_0 + 1$ \\~\\

We know at the start of the iteration, for all natural numbers $k$ in between 0 and \texttt{len(nums\_so\_far[$i_0$])} inclusive, if k is coprime to 2 and coprime to 3, then $k$ in \texttt{len(nums\_so\_far)}. \\
By loop invariant 1, we know $k$ can be written as $k = 1 + 6e$ or $k = 5 + 6e$ where $e$ is some $\Z$ ($e$ does not need to be the same for every instance of $k$). \\~\\

Case 1: \texttt{nums\_so\_far[$i_0$]} $= 1 + 6e$, where $ e \in \Z$ \\
-In this case, by loop invariant 1 we know at the end of the iteration, \texttt{nums\_so\_far[$i$]} $= 5 + 6e$ \\
-We must verify that for all natural numbers $k$ in between $2 + 6e$ and $5 + 6e$ inclusive, the implication \\ $\gcd(k, 2) = 1 \wedge \gcd(k, 3) = 1 \implies k \in$ \texttt{nums\_so\_far} must hold \\
-We know that of the numbers in the range above, only $5 + 6e$ is in \texttt{nums\_so\_far} \\
$1 + 6e + 1 = 2 + 6e = 2(1 + 3e)$ thus, $2|2 + 6e$ and $2|2$ and so $\gcd(2 + 6e, 2) \neq 1$ \\
$1 + 6e + 2 = 3 + 6e = 3(1 + 2e)$ thus, $3|3 + 6e$ and $3|3$ and so $\gcd(3 + 6e, 3) \neq 1$ \\
$1 + 6e + 3 = 4 + 6e = 2(2 + 3e)$ thus, $2|4 + 6e$ and $2|2$ and so $\gcd(4 + 6e, 2) \neq 1$ \\
$1 + 6e + 4 = 5 + 6e$ by loop invariant 1, we know $gcd(5 + 6e, 2) = 1 \wedge gcd(5 + 6e, 3) = 1$ \\
$5 + 6e$ is the only number of the above in \texttt{nums\_so\_far}, thus satisfying the inequality for all natural numbers $k$ in between 0 and \texttt{len(nums\_so\_far[$i$])} inclusive by the end of the iteration\\~\\

Case 2: \texttt{nums\_so\_far[$i_0$]} $= 5 + 6e$, where $ e \in \Z$ \\
-In this case, by loop invariant 1 we know at the end of the iteration, \texttt{nums\_so\_far[$i$]} $= 1 + 6(e + 1)$ \\
-We must verify that for all natural numbers $k$ in between $6 + 6e$ and $1 + 6(e + 1)$ inclusive, the implication \\ $\gcd(k, 2) = 1 \wedge \gcd(k, 3) = 1 \implies k \in$ \texttt{nums\_so\_far} must hold \\
-We know that of the numbers in the range above, only $1 + 6(e + 1)$ is in \texttt{nums\_so\_far} \\
$5 + 6e + 1 = 6 + 6e = 2(3 + 3e)$ thus, $2|6 + 6e$ and $2|2$ and so $\gcd(6 + 6e, 2) \neq 1$ \\
$5 + 6e + 2 = 7 + 6e = 1 + 6(e + 1)$ by loop invariant 1, we know $gcd(1 + 6(e + 1), 2) = 1 \wedge gcd(1 + 6(e + 1), 3) = 1$ \\
$1 + 6(e + 1)$ is the only number of the above in \texttt{nums\_so\_far}, thus satisfying the inequality for all natural numbers $k$ in between 0 and \texttt{len(nums\_so\_far[$i$])} inclusive by the end of the iteration \\~\\

Therefore at the end of the iteration, for all natural numbers $k$ in between 0 and \texttt{len(nums\_so\_far[$i$])} inclusive, if k is coprime to 2 and coprime to 3, then $k$ in \texttt{len(nums\_so\_far)}.
\end{proof}
\end{enumerate}

\item[4.]
Complete this part in the provided \texttt{a4\_part2.py} starter file.
Do \textbf{not} include your solution in this file.

\item[5.]
Complete this part in the provided \texttt{a4\_part2.py} starter file.
Do \textbf{not} include your solution in this file.
\end{enumerate}

\newpage

\section*{Part 3: Running-Time Analysis}

\begin{enumerate}
\item[1.]
-Let $n$ be the input integer $n$ \\
-Let $k$ be the number of iterations the loop has ran \\
-Let $i_k$ be the value of \texttt{nums\_so\_far[$i$]} $+ ~6$ on the $k$-th iteration \\~\\

The cost of the assignment statement at the start is constant time, it takes 1 step. \\~\\

To analyze the while loop, we need to determine the cost of each iteration and the total number of iterations. \\
-Each iteration has 2 constant time statements, so we'll count that as one step. \\
-To find the number of iterations, we need to find the smallest value of $k$ such that $i_k \geq n$ (making the loop condition false). There are two possible formulas for $i_k$: \\
$~~~~~~~$ if $k$ is odd, $i_k = 1 + 6(\frac{k + 1}{2})$ \\
$~~~~~~~$ if $k$ is even, $i_k = 5 + 6(\frac{k}{2})$ \\
-To find how many times the loop iterates at most, use the formula that will always be less than the other, thus requiring a greater $k$ value to satisfy the inequality (we want to find the greater upper bound). In this case, it's the first formula. Then isolate for $k$ in the inequality $i_k \geq n$\\
$~~~~~~~~ i_k \geq n$ \\
$~~~~~~~~ 1 + 6(\frac{k + 1}{2}) \geq n$ \\
$~~~~~~~~ \frac{k + 1}{2} \geq \frac{n - 1}{6}$ \\
$~~~~~~~~ k \geq \frac{n - 1}{3} - 1$ \\
-We need to find the smallest value of k such that $k \geq \frac{n - 1}{3} - 1$, which is the definition of a ceiling function, and so the smallest value of $k$ can be expressed as $k \geq \lceil\frac{n - 1}{3} - 1\rceil$
We know the loop runs at most $\lceil\frac{n - 1}{3} - 1\rceil$ times, with one step per iteration, for a total of $\lceil\frac{n - 1}{3} - 1\rceil$ steps. \\~\\

The cost of the return statement is constant time, it takes 1 step. \\~\\

Putting it all together, the function \texttt{coprime\_to\_2\_and\_3} has a running time of \\
$1 + \lceil\frac{n - 1}{3} - 1\rceil + 1 = \lceil\frac{n - 1}{3} - 1\rceil + 2$ which is $\cO(n)$

\item[2.]
-Let $P$ be the size of the input set \texttt{primes}  \\
-Let $m$ be the product of the numbers in \texttt{primes} \\

The cost of the first assignment statement is constant time, it takes 1 step. \\~\\

The second assignment statement have a running time of n steps, where n is the size of the input. So in this case, it takes P steps. \\~\\

To analyze the loop, we will first determine the number of steps in the inner loop and then the number of steps in the outer loop. \\
-The inner loop contains one constant time statement, so we'll count that as one step per iteration of the inner loop. \\
-The inner loop iterates for each element in \texttt{primes}, so it iterates $P$ times with 1 step per iteration. \\
-The outer loop contains 2 constant time statements and the inner loop that iterates $P$ times, so we'll count it as taking $P + 1$ steps. \\
-The outer loop runs one less time than the product of the numbers in \texttt{primes}, so it iterates $m -1$ times, with $P + 1$ steps per iteration; for a total of $(m - 1) \cdot (P + 1)$. \\~\\

The cost of the return statement is constant time, it takes 1 step. \\~\\

Putting it all together, the function \texttt{starting\_coprime\_numbers} has a running time of \\
$1 + P + (m - 1) \cdot (P + 1) + 1 = (m - 1) \cdot (P + 1) + P + 2$ which is $\Theta(m \cdot P)$

\item[3.]
-Let $P$ be the size of the input set \texttt{primes}  \\
-Let $m$ be the product of the numbers in \texttt{primes} \\
-Let $n$ be the input integer $n$ \\
-Let $k$ be the number of iterations the loop has ran \\
-Let $i_k$ be the value of \texttt{nums\_so\_far[$i$]} $+ ~6$ on the $k$-th iteration \\~\\

The first assignment statement is a function call to \texttt{starting\_coprime\_numbers}, which as established previously, takes $(m - 1) \cdot (P + 1) + P + 2$ steps. \\~\\

The second assignment statement is constant time, it takes 1 step. \\~\\

The third assignment statement have a running time of n steps, where n is the size of the input. So in this case, it takes $P$ steps. \\~\\

To analyze the while loop, we need to determine the cost of each iteration and the total number of iterations. \\
-Each iteration has 2 constant time statements, so we'll count that as one step. \\
-To find the number of iterations, we need to find the smallest value of $k$ such that $i_k \geq n$ (making the loop condition false). Using the same thought process from question 1 to find the formula that gives us the greatest $k$-value to find the greater upper bound between all the possible formulas. The formula is this case is:\\
$~~~~~~~~ 1 + m(1 + \lfloor\frac{k}{\phi(m)}\rfloor)$ \\
-$\phi$ here is representing Euler's Totient Function
-To find how many times the loop iterates at most, isolate for $k$ in the inequality $i_k \geq n$\\
$~~~~~~~~ i_k \geq n$ \\
$~~~~~~~~ 1 + m(1 + \lfloor\frac{k}{\phi(m)}\rfloor) \geq n$ \\
$~~~~~~~~ 1 + \lfloor\frac{k}{\phi(m)}\rfloor \geq \frac{n - 1}{m}$ \\
$~~~~~~~~ \frac{k}{\phi(m)} \geq \lceil\frac{n - 1}{m} + 1\rceil$ \\
$~~~~~~~~ k \geq \phi(m)\lceil\frac{n - 1}{m} + 1\rceil$ \\
We know the loop runs at most $\phi(m)\lceil\frac{n - 1}{m} + 1\rceil$ times, with one step per iteration, for a total of $\phi(m)\lceil\frac{n - 1}{m} + 1\rceil$ steps. \\~\\

The cost of the return statement is constant time, it takes 1 step. \\~\\

Putting it all together, the function \texttt{coprime\_to\_all} has a running time of \\
$(m - 1) \cdot (P + 1) + P + 2 + 1 + P + \phi(m)\lceil\frac{n - 1}{m} + 1\rceil + 1 = (m - 1) \cdot (P + 1) + \phi(m)\lceil\frac{n - 1}{m} + 1\rceil + 2P + 4$ \\
which is $\cO(m \cdot P + \frac{n}{m})$
\end{enumerate}

\section*{Part 4: Two New Cryptosystems}

Complete this part in the provided \texttt{a4\_part4.py} starter file.
Do \textbf{not} include your solution in this file.

\end{document}
