\documentclass[fontsize=11pt]{article}
\usepackage{amsfonts}
\usepackage{amsmath}
\usepackage{amsthm}
\usepackage[utf8]{inputenc}
\usepackage[margin=0.75in]{geometry}

\title{CSC110 Assignment 4: Number Theory, Cryptography, and Algorithm Running Time}
\author{TODO: FILL IN YOUR NAME HERE}
\date{\today}

% Some useful LaTeX commands. You are free to use these or not, and also add your own.
\newcommand{\N}{\mathbb{N}}
\newcommand{\Z}{\mathbb{Z}}
\newcommand{\R}{\mathbb{R}}
\newcommand{\cO}{\mathcal{O}}
\newcommand{\floor}[1]{\left\lfloor #1 \right\rfloor}

\begin{document}
\maketitle

\section*{Part 1: Practice with Proofs}

\begin{enumerate}

\item[1.] Statement to prove:

$$\forall a, k, n \in \Z,~ \gcd(a, n) = 1 \Rightarrow \gcd(a + kn, n) = 1$$

\begin{proof}
TODO: Your proof goes here.
\end{proof}

\item[2.] Statement to prove (we've expanded the definition of Omega for you!):

$$\exists c, n_0 \in \R^+,~ \forall n \in \N,~ n \geq n_0 \Rightarrow \log_{3} n - \log_{11} n \geq c \cdot \log_{14} n$$

\begin{proof}
TODO: Your proof goes here.
\end{proof}

\item[3.] Statement to prove (we haven't expanded the definition of Big-O for you, but we encourage you to do so yourself):

$$\forall f, g: \N \to \R^{\geq 0},~ g \in \cO(f) \land \big(\forall m \in \N,~ f(m) \geq 1 \big) \Rightarrow g \in \cO(\floor{f})$$

\begin{proof}
TODO: Your proof goes here.
\end{proof}

\end{enumerate}

\newpage

\section*{Part 2: Generating Coprime Numbers}

\begin{enumerate}

\item[1.]
Not to be handed in.

\item[2.]
Complete this part in the provided \texttt{a4\_part2.py} starter file.
Do \textbf{not} include your solution in this file.

\item[3.]
Prove that each loop invariant holds.

\begin{enumerate}
\item[a.] Loop Invariant 1
\begin{proof}
TODO: Your proof goes here.
\end{proof}

\item[b.] Loop Invariant 2
\begin{proof}
TODO: Your proof goes here.
\end{proof}

\item[c.] Loop Invariant 3
\begin{proof}
TODO: Your proof goes here.
\end{proof}

\item[d.] Loop Invariant 4
\begin{proof}
TODO: Your proof goes here.
\end{proof}
\end{enumerate}

\item[4.]
Complete this part in the provided \texttt{a4\_part2.py} starter file.
Do \textbf{not} include your solution in this file.

\item[5.]
Complete this part in the provided \texttt{a4\_part2.py} starter file.
Do \textbf{not} include your solution in this file.
\end{enumerate}

\newpage

\section*{Part 3: Running-Time Analysis}

\begin{enumerate}
\item[1.]
TODO: Running-time analysis of \texttt{coprime\_to\_2\_and\_3}.

\item[2.]
TODO: Running-time analysis of \texttt{starting\_coprime\_numbers}.

\item[3.]
TODO: Running-time analysis of \texttt{coprime\_to\_all}.
\end{enumerate}

\section*{Part 4: Two New Cryptosystems}

Complete this part in the provided \texttt{a4\_part4.py} starter file.
Do \textbf{not} include your solution in this file.

\end{document}
